Podczas implementacji systemu udało się większość stawianych mu wymagań funkcjonalnych i niefunkcjonalnych. Specyfika tematu, a w szczególności jego otwartość na rozwój i skalowanie powoduje, że każdą ze zrealizowanych funkcjonalności można jeszcze dalej rozwijać oraz dodawać nowe praktycznie bez ograniczeń. Mówiąc najogólniej: spełnione zostało podstawowe kryterium, o którym wspomniano w poprzednim rozdziale, tj. otrzymaliśmy w pełni działający system, który potwierdza naszą tezę i wykonuje zadanie, jakie mu powierzono.

\section{Zarys wymagań}
Pierwszym i podstawowym wymaganiem stawianym systemowi jest sprawne i bezbłędne przetwarzanie danych. Potrzeba szybkości nie podlega dyskusji -- projektując systemy przetwarzania danych jest to zawsze jednym z podstawowych kryteriów oceny systemu. Bezbłędność jest kolejnym niezwykle ważnym kryterium: podczas dość skomplikowanej ścieżki, jaką pokonują dane w naszym programie jest wiele miejsc, w których dane mogą ulec przekłamaniu lub może wkraść się do nich błąd.

Tak więc po pierwsze potrzebujemy kompilatora, który produkuje poprawne programy. Niezwykle łatwo przy budowie języka przeoczyć pewne własności, które ujawniają się dopiero przy pewnych szczególnych sytuacjach (jak na przykład pętla o zerowej ilości iteracji). Przez poprawność programów rozumiemy nie tylko poprawność logiczną kodu maszynowego, ale również odfiltrowanie niepoprawnych programów użytkownika. Po drugie, transmisja danych musi być niezawodna. Przy niskopoziomowym przesyłaniu danych przez port szeregowy łatwo może dojść do przekłamania niektórych bitów i w efekcie do zepsucia całego wysyłanego do koprocesora programu. Aby tego uniknąć, należy zadbać o właściwą obsługę portu od strony oprogramowania, układu FPGA, jak i sprzętu użytego do połączenia całego systemu. Kolejnym elementem warunkującym poprawność działania systemu jest właściwa implementacja układu na FPGA, gdzie łatwo o subtelne pomyłki, jak na przykład za mała liczba cykli przeznaczanych na odczyt z pamięci.

Możemy zatem wyróżnić w naszym systemie wymagania funkcjonalne:
\begin{itemize}
  \item stworzenie języka programowania, wspierającego m.in. następujące operacje:
  \begin{itemize}
    \item deklaracje i przypisania zmiennych
    \item instrukcje warunkowe
    \item pętle
    \item kontrolę typów
    \item operacje arytmetyczne na wektorach
  \end{itemize}
  \item kompilacja programów w wyżej opisanym języku do postaci binarnej
  \item przesyłanie programów do procesora na układzie FPGA
  \item odbieranie wyników z procesora
  \item wykonanie programu na procesorze (operacje na krótkich wektorach powinny być atomowe i wykonywać się w stałej, niewielkiej liczbie cykli)
\end{itemize}

Dodatkowo, możemy mówić o (wspomnianych wyżej) wymaganiach niefunkcjonalnych:
\begin{itemize}
  \item ,,wysokopoziomowość'' języka -- użytkownik nie powinien pisać ręcznie typowych działań na wektorach
  \item wygoda przetwarzania danych wektorowych
  \item wydajność całego systemu
  \item niezawodność i brak błędów (istone zwłaszcza przy implementacji komunikacji między komputerem a układem FPGA)
\end{itemize}

\section{Zrealizowane funkcjonalności}
Wyniki przeprowadzanych testów wskazują, że system spełnia powyższe wymagania. Wydajność systemu gwarantowana jest przez równoległość przetwarzania danych po stronie koprocesora, oraz odpowiedni dobór technologii i przemyślany program po stronie kompilatora (użyto języka Haskell, który świetnie nadaje się do tworzenia kompilatorów, a dodatkowo posiada niezwykle dopracowany, optymalizujący kompilator GHC).


Funkjonalności dostarczane przez system to:
\begin{itemize}
  \item kompilacja programów napisanych w dedykowanym języku programowania, ukierunkowanym na przetwarzanie danych wektorowych. Język wspiera natywnie operacje arytmetyczne na wektorach, takie jak: dodawanie, odejmowanie, mnożenie (skalarne), dzielenie (element-przez-element) oraz posiada mechanizm statycznego sprawdzania długości wektorów. Kompilacja przebiega kilkuetapowo i składa się między innymi z fazy parsowania, sprawdzania typów (typecheckingu), generacji kodu assemblera i binaryzacji.
  \item własny język assembler, przystosowany do architektury koprocesora, uwzględniający wektorową naturę języka.
  \item wykonywanie programów na koprocesorze, zaimplementowanym na układzie FPGA. Koprocesor ma architekturę wektorową, co oznacza, że wektorowe operacje na wektorach stałej długości są wykonywane w jednym cyklu zegara, a większe wektory są rozbijane na mniejsze -- takie, które procesor potrafi obsłużyć w jednym cyklu.
  \item przesyłanie skompilowanych programów w postaci binarnej z komputera na koprocesor i z powrotem za pomocą portu UART (transmisja szeregowa).
\end{itemize}
