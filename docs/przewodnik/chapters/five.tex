Podczas implementacji projektu udało się zrealizować w pełni stawiany sobie cel, tj. uzyskać system, który działa bezbłędnie i dowodzi możliwości zastosowania takiej technologii w dziedzinach wymagających wydajnego, a jednocześnie przyjaznego dla programisty przetwarzania danych. Jako rezultat uzyskaliśmy język programowania, assembler i kompilator, który pozwalaja na tłumaczenie programów do assemblera, a następnie do kodu maszynowego (który również posiada opracowany przez nas, dostosowany do naszych potrzeb format). W języku zaimplementowano sposób zrównoleglania obliczeń prowadzonych na wektorach przez podział wektora na małe podwektory o stałej długości, które następnie są przetwarzane w jednym cyklu przez jednostkę arytmetyczno-logiczną (każdy z kawałków wektora jest już przetwarzany sekwencyjnie, używając do tego celu dodatkowych stosów).

Oprócz tego powstał koprocesor, który potrafi wykonywać każdą instrukcję języka, a co za tym idzie wszystkie poprawne programy w nim napisane (z uwzględnieniem jedynie górnego ograniczenia na rozmiar programu, spowodowanego fizyczną ilością zasobów w koprocesorze). Koprocesor ma architekturę ukierunkowaną na przetwarzanie wektorów, co realizowane jest przez obsługę równoległości opisaną przy okazji opisu mechanizmów zrównoleglania obliczeń zaimplementowanych w języku programowania.
