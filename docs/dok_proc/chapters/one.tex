Prace nad projektem prowadzone były zgodnie z przyrostowym modelem wytwarzania oprogramowania. Cały proces twórczy podzielić można na 4 etapy - z których każdy rozpoczynał się spotkaniem z opiekunem projektu(dokładniej opisane w następnym rozdziale). Początkowe etapy miały charakter badawczy i wymagały podjęcia istotnych decyzji co do działania platformy. Kolejne etapy dostarczały coraz to nowe elementy całego systemu. I tak kolejno powstawały: drzewo syntaktyczne, parser, system sprawdzania poprawności typów, translator do asemblera, moduł binaryzacji asemblera, moduł komunikacji z FPGA oraz w końcu - również przyrostowo - sam procesor.

\subsection{Role i podział prac}
Szczegółowy podział ról i prac opisany jest w przewodniku po pracy. Opisując metodykę wspomnieć należy, że cała koncepcja systemu powstała w wyniku wspólnych rozważań i analiz obu twórców, natomiast implementacja każdej części kodu była procesem w którym obaj autorzy stale współpracowali, wzajemnie poprawiając i testując implementowane przez siebie fragmenty.

\subsection{Narządzia wykorzystane przy realizacji projektu}
Całość kodu i dokumentacji projektowej przechowywana była na platformie \textbf{GitHub}. Umożliwiało to łatwą analize zmian wprowadzonych przez poszczególnych użytkowników oraz wersjonowanie. Do przechowywania notatek o aktualnych i przyszłych zadaniach wykorzystaliśmy platformę \textbf{Trello}. Ustalenia, wizje i koncepcje ze spotkań i narad pomiędzy autorami edytowaliśmy wspólnie przy użyciu \textbf{GoogleDocs}.